% ****** Start of file aipsamp.tex ******
%
%   This file is part of the AIP files in the AIP distribution for REVTeX 4.
%   Version 4.1 of REVTeX, October 2009
%
%   Copyright (c) 2009 American Institute of Physics.
%
%   See the AIP README file for restrictions and more information.
%
% TeX'ing this file requires that you have AMS-LaTeX 2.0 installed
% as well as the rest of the prerequisites for REVTeX 4.1
%
% It also requires running BibTeX. The commands are as follows:
%
%  1)  latex  aipsamp
%  2)  bibtex aipsamp
%  3)  latex  aipsamp
%  4)  latex  aipsamp
%
% Use this file as a source of example code for your aip document.
% Use the file aiptemplate.tex as a template for your document.
\documentclass[%
 aip,
%jmp,%
%bmf,%
%sd,%
rsi,%
 amsmath,amssymb,
%preprint,%
 reprint,%
%author-year,%
%author-numerical,%
]{revtex4-1}

\usepackage{graphicx}% Include figure files
\usepackage{dcolumn}% Align table columns on decimal point
\usepackage{bm}% bold math
%\usepackage[mathlines]{lineno}% Enable numbering of text and display math
%\linenumbers\relax % Commence numbering lines

\usepackage{xcolor}

\usepackage{todonotes}

\begin{document}

\preprint{AIP/123-QED}

\title[Convergence in expanded ensemble simulations]{Understanding and ameliorating sampling bottlenecks in expanded ensemble free energy simulations}% Force line breaks with \\My greatest achievement was getting

% \title[Improved adaptive seeding]{Improved adaptive estimates of MSMs built from short reseeding trajectories}% Force line breaks with \\

%\title[Sample title]{On the use of short reseeding trajectories to sample Markov state models}% Force line breaks with \\

% \author{A. Author}
%  \altaffiliation[Also at ]{Physics Department, XYZ University.}%Lines break automatically or can be forced with \\
% \author{B. Author}%
%  \email{Second.Author@institution.edu.}
% \affiliation{ 
% Authors' institution and/or address%\\This line break forced with \textbackslash\textbackslash
% }%

\author{Vincent A. Voelz}
\email{voelz@temple.edu}
\affiliation{Department of Chemistry, Temple University, Philadelphia, PA 19122, USA}


\date{\today}% It is always \today, today,
             %  but any date may be explicitly specified

\begin{abstract}
\color{red}VAV: needs  updating. Expanded ensemble algorithms can sample multiple thermodynamic states in a single-replica simulation to estimate free energies, and provide easily interpretable results.  These advantages are offset by poor convergence issues that can arise for certain alchemical transformations.  Here, we show that bottlenecks in sampling arise when the distribution of energy differences between thermodynamic ensembles becomes too broad.  To ameliorate this problem, we develop a simple method to optimize the schedule of alchemical intermediates after some initial sampling.  We also develop a simple model of sampling hysteresis that results from the exponential scaling of the bias increment in the Wang-Landau algorithm, and how a power-law scaling of this increment can help avoid hysteresis.   These model enable the formulation of simple guidelines to help delay the onset of sampling bottlenecks and hysteresis.  We apply these techniques to example systems including  double-decoupling absolute free energy simulations to estimate ligand binding, and alchemical free energy simulations to estimate relative free energies of mutations.\color{black}
\end{abstract}

%\pacs{Valid PACS appear here}% PACS, the Physics and Astronomy
                             % Classification Scheme.
\keywords{alchemical free energy perturbation, flat-histogram methods, ligand binding, Monte Carlo}%Use showkeys class option if keyword
                              %display desired
\maketitle

% \begin{quotation}
% The ``lead paragraph'' is encapsulated with the \LaTeX\ 
% \verb+quotation+ environment and is formatted as a single paragraph before the first section heading. 
% (The \verb+quotation+ environment reverts to its usual meaning after the first sectioning command.) 
% Note that numbered references are allowed in the lead paragraph.
% %
% The lead paragraph will only be found in an article being prepared for the journal \textit{Chaos}.
% \end{quotation}

\section*{Introduction}
% In the last few years, there has been great interest in using expanded-ensemble (EE) methods\cite{lyubartsev1992new} to perform absolute and relative ligand binding free energy calculations (CITE Shirts papers, others).\cite{monroe2014converging}

% In the recent SAMPL6 challenge to predict host-guest affinities, EE methods performed with similar accuracy and efficiency to Hamiltonian replica exchange (HREX), with the advantage that the calculation only requires a single simulation that does not have to be coupled to other replicas.\cite{rizzi2020sampl6}   This property is highly desirable for large-scale \textit{in silico} screening efforts performed asynchronously using massively parallel distributed computing platforms like Folding@home, as pursued recently by Hurley et al.

Expanded ensemble algorithms allow multiple thermodynamic ensembles to be sampled in a single simulation.   This may be particularly useful for large-scale virtual screening on distributed computing platforms.

EE methods perform well at predicting absolute binding free energies of ligald (in SAMPL challenges), and relative binding free energies for small-molecule inhbitors (Si Zhang et al).   But EE method have not seen widespread adoption, in part because of sampling problems that often arise in practice.

One type of problem commonly seen in EE approaches are bottlenecks in sampling the full range of the alchemical intermediates.  Such bottlenecks are a result of poor acceptance of transitions between thermodynamic ensembles, an issue which is related to, but distinct from, the overlap in phase space between the thermodynamic ensembles. 

Using the standard Wang-Landau flat-histogram algorithm as implemented in the GROMACS molecular dynamics simulation software, we have noticed that once these sampling bottlenecks begin, they become progressively worse, preventing visitation to all alchemical intermediates.

In this work, we develop a simple model based on 1D harmonic potentials that explains the appearance of sampling bottlenecks, and how they lead to hysteresis that further impedes efficient exploration of alchemical intermediates.  We also use the simple model to derive an algorithm that can help refine the schedule of alchemical intermediates to help avoid bottlenecks. 


\subsection*{The expanded ensemble method}

In traditional free energy perturbation approaches, a number of simulations are performed in different thermodynamic ensembles indexed by $\lambda$, corresponding to (reduced) potential energy functions $u_{\lambda}(x) = \beta U_{\lambda}(x)$, where $\beta = 1/kT$.  For instance, to calculate a molecule's free energy of solvation, the $\lambda$ parameter might interpolate between 0 and 1 for potential energy functions $u_{\lambda}(x) = u_{\text{bonded}}(x) + \lambda u_{\text{nonbonded}}(x)$, for a simulation of the molecule in solution.  Due to the statistical inefficiencies of exponential averaging, it is very difficult to directly estimate the free energy $f = -\ln Z_{\lambda=1}/Z_{\lambda=0}$, where $Z_{\lambda} = \int e^{-u_{\lambda}(x)} dx$ is the partition function.  Instead,  the calculation is performed at many intermediate $\lambda$ values (e.g. $\lambda_i = 0, 0.05, 0.1, .... 1$), with the energies computed from these simulations used as input to one of several multi-ensemble free energy estimators.\cite{shirts2008statistically, TI} 

Consider a set of thermodynamic ensembles defined by reduced potential energy functions $u_i(x) = \beta U(x|\lambda_i)$, for a set of $\lambda$ values indexed by $i$, where $\lambda$ ranges between 0 and 1, and $\beta = 1/k_BT$.  The goal of expanded ensemble (EE) methods is to use Monte Carlo sampling to perform a random walk in $\lambda$-space, throughout a single simulation.  This is achieved through the use of constant bias potentials $\tilde{f}_i$, that modify each potential energy function as $u_i'(x) = u_i(x) - \tilde{f}_i$. The values of $\tilde{f}_i$ are adjusted on-the-fly, and iteratively refined to achieve Monte Carlo transition probabilities $p_{ij}$ between thermodynamic ensembles $i$ and $j$ that are equal in both directions ($p_{ij} = p_{ji}$).  If this can be achieved, then the difference in the biases $\Delta \tilde{f}_{ij} = \tilde{f}_j - \tilde{f}_i$ are equal to the true difference in free energies $\Delta f_{ij}$ (see Appendix A).

In practice, bottlenecks often arise in the Monte Carlo sampling that lead to poor convergence of the free energy estimates.  One source of sampling bottlenecks comes from so-called ``hidden'' barriers in the free energy landscape, often linked to slow conformational changes of a ligand or receptor.\cite{slow-barriers}.  If the slow degrees of freedom can be identified, then appropriate enhanced sampling schemes can be employed.  But because such bottlenecks are highly system-dependent,  it is difficult to overcome sampling barriers of these types in a general way, and past efforts have addressed such problems on a case-by-case basis.\cite{case-by-case}

Another source of bottlenecks arises from poor Monte Carlo acceptance probabilities, and these can be addressed in a much more general way. Although such sampling barriers may be hard to predict ahead of time, they can easily be identified using short simulations, and steps can be taken to adjust the parameters of the expanded ensemble algorithm, or introduce more lambda intermediates.

Our goal in this study is to demonstrate, using a very simple system of 1D harmonic oscillators, with no hidden barriers, the general phenomenon that Monte Carlo bottlenecks in Wang-Landau expanded ensemble sampling arise from broad distributions $P(\Delta u_{ij})$, leading to hysteresis in $\lambda$-sampling.   Sampling bottlenecks can be alleviated by introducing more thermodynamic ensembles at intermediate values of $\lambda$, but not so many as to create additional travel time.

To develop some simple rules for how to choose the best algorithm parameters, we construct a simple continuous-time kinetic model of the evolution of free energy estimates, which predicts conditions under which hysteresis occurs.  \color{red} This model suggests that $\lambda$ intermediates be chosen to achieve at least $\sim XX\%$ acceptance probability between neighboring ensembles .... VAV:  I don't know what this will say yet. :) \color{black}

%%%%%%%%%%%%%%%%%%%%%%%%%%
\section*{Methods}

\subsection{The Wang-Landau algorithm}




%%%%%%%%%%%%%%%%%%%%%%%%%%
\section*{Results}

\subsection*{A simple model of expanded ensemble sampling bottlenecks}




\section*{Discussion}

Markov State Model approaches ...

\begin{acknowledgments}
The authors thank the participants of Folding@home, without whom this work would not be possible.  We thank D. E. Shaw Research for providing access to folding trajectory data. This research was supported in part by the National Science Foundation through major research instrumentation grant number CNS-09-58854 and National Institutes of Health grants 1R01GM123296 and NIH Research Resource Computer Cluster Grant S10-OD020095.
\end{acknowledgments}

%
% ****** End of file aipsamp.tex ******


% \nocite{*}
\bibliography{toy-ee}% Produces the bibliography via BibTeX.


\section*{Appendix}

\subsection{Forward and backward Monte Carlo transition probabilities are equal when expanded ensemble biases are the true free energies}

Consider the free energy between two thermodynamic ensembles, indexed by $i$ and $j$.  The exponential of the negative free energy difference $\Delta f_{ij} = -\ln Z_j/Z_i$ can be expressed
\begin{equation}
      e^{-\Delta f_{ij}} = \frac{\int e^{-u_j(x)} dx}{\int e^{-u_i(x)} dx}  
            = \frac{\int e^{-(u_j(x)-u_i(x))}e^{-u_i(x)} dx}{\int e^{-u_i(x)} dx}.
\end{equation}
Therefore, the free energy is
\begin{equation}
    \Delta f_{ij} = - \ln \langle e^{-\Delta u_{ij}} \rangle_i,
\end{equation}
where $\Delta u_{ij} = u_j(x)-u_i(x)$ and the bracketed term denotes the thermodynamic average over ensemble $i$. This is known as the Zwanzig relation.\cite{zwanzig1954high} Similar reasoning leads to 
\begin{equation}
    \Delta f_{ij} = -\Delta f_{ji} =  + \ln \langle e^{-\Delta u_{ji}} \rangle_j.
\end{equation}

Next, consider biased ensembles $u_i'(x) = u_i(x) - \tilde{f}_i$ where we set each bias $\tilde{f}_i$ to the true free energy $f_i$, such that $\Delta u_{ij}'(x) = \Delta u_{ji}(x) - \Delta f_{ji}$.  By the Zwanzig relation, the free energy differences $\Delta f_{ij}'$ between any two biased ensembles is zero:
\begin{align}
    \Delta f_{ij}' &= - \ln \langle e^{-\Delta u_{ij}'(x)} \rangle_i \\
    &= - \ln \langle e^{-\Delta u_{ij}(x)  + \Delta \tilde{f}_{ij} } \rangle_i \\
    &= - \ln \langle e^{-\Delta u_{ij}(x)}\rangle_i - \Delta \tilde{f}_{ij}  = 0.
\end{align}

Thermodynamic ensembles biased in this way also have equal Monte Carlo acceptance probabilities $p_{ij}$ and $p_{ji}$.  Assuming a simple Metropolis criterion for accepting a Monte Carlo move, the ensemble-average acceptance of a move $i \rightarrow j$ is:
\begin{align}
    p_{ij} &= \frac{ \int \min [ 1, e^{-\Delta u_{ij}(x) + \Delta f_{ij} } ] e^{-u_i(x)} dx} {Z_1} \\
    &= \frac{\int_{\Delta u_{ij} < \Delta f_{ij} } (1) e^{-u_i(x)} dx}{Z_1} + \frac{\int_{\Delta u_{ij} \geq \Delta f_{ij} } e^{-u_j(x)} e^{\Delta f_{ij}} dx}{Z_1} \\
    &=  \frac{Z_1}{Z_2} \frac{\int_{\Delta u_{ji} > \Delta f_{ji} } e^{-u_i(x)} e^{\Delta f_{ji}} dx}{Z_1} + \frac{Z_1}{Z_2} \frac{\int_{\Delta u_{ij} \geq \tilde{f}_{ij} } e^{-u_j(x)} dx}{Z_1} \\
    &=  \frac{\int_{\Delta u_{ji} > \Delta f_{ji} } e^{-\Delta u_{ji}(x)} e^{-u_j(x)} dx}{Z_2} + \frac{\int_{\Delta u_{ji} \leq \Delta \tilde{f}_{ji} } (1) e^{-u_j(x)} dx}{Z_2} \\
    &= \frac{ \int \min [ 1, e^{-\Delta u_{ji}(x) + \Delta f_{ji} } ] e^{-u_j(x)} dx} {Z_2} = p_{ji}. \\
\end{align}




\newpage


% \section*{Supporting Information}
% \subsubsection*{Supporting Figures}



\end{document}